\documentclass[12pt]{article}
\usepackage[utf8x]{inputenc}
\usepackage{graphicx,linguex,natbib,aleks,amsmath,amsthm}
\bibpunct{(}{)}{;}{a}{,}{,}
\bibliographystyle{plainnat}
\author{Magdalena Leshtanska \and Aleksandar Dimitrov}
\title{Error-Centric Annotation of Learner Corpora}
\begin{document}
\maketitle

\newtheorem{thm}{Theorem}[section]
\newtheorem{definition}[thm]{Definition}

\tableofcontents
\abstract{This paper presents an annotation method for errors in learner
language that strives to be as compact and precise as possible, without
sacrificing on generality or correctness. We develop contextualisation
and underspecification methods for error annotations, based on a flexible annotation-driven markup.}

\section{Introduction}\label{sec:intro}

Recently, there has been a growing interest in learner language corpora and
learner errors analysis \cite{nessel}. There is a need to identify errors in
text produced by second language learners, to make the automatic processing of
those texts feasible (ibid.). Also, the potential of learner errors for
educational purposes and language acquisition insight has become increasingly
recognized (ibid.), hence, many annotation schemes have been devised to
identify and classify learner errors, such as \cite{negrillo}, \cite{CLC},
\cite{ludeling}

We propose a novel approach to annotating learner errors, based on the idea that
an error is an inconsistency with its surroundings. This basic markup approach
can be  combined with a taxonomy for error classification. We
have coupled it with learner error  classification system and used the resulting
scheme to annotate a subset of the NOCE corpus (\cite{negrilloph}), a corpus of
beginning to intermediate English learners of Spanish.

Section \ref{sec:scheme} will introduce the general idea behind our annotation
scheme and elaborate on more technical details. Section 4 gives a brief overview
of real world performance of our method. An annotation manual is contained in
Appendix A.

\section{Error Taxonomy and Annotation Scheme}\label{sec:scheme}

Devising an exhaustive taxonomy of errors that can appear in natural language
seems a daunting task, since science has so far failed to regularize what
\textit{is} a valid utterance of a language.  Therefore, our annotation scheme
strives  to  do some things only, and do them well.

\subsection{Basic Concepts}\label{sec:threeholies}

Our annotation scheme presents three distinct kinds of errors:

\begin{itemize}
  \item \textit{Grammatical context.} each introduced token enforces constraints
  on the utterance, or on parts of it. In \ref{swim}, \textit{remember} requires
  a gerund, thus the form \textit{swim} is ungrammatical in this context. We
  call such mistakes \textbf{grammar errors}.

  \ex. *I remember swim in the river\label{swim}

  \item \textit{Semantic context.}  In \ref{glue}, though the grammatical
  structure is correct, the predicate does not fit the contextual information.
  This is a contextual error.

  \ex. *Yesterday I will go to the fair.\label{glue}

  \item \textit{Spelling errors.} Spelling errors are typically not influenced
  by contextual information, except for the normative context of a given
  language's orthographic rules. Punctuation is a special case, since it may
  also carry semantic information.

  \end{itemize}

\subsection{Calling Errors by Name}\label{sec:taxonomy}

The classification of errors, as well as establishing an error's scope and 
influence on the rest of the sentence are the main tasks in the annotation
process of learner language. The following section documents our approach to
these problems.

Firstly, every annotation scheme that operates directly on the text or spans
over a piece of text will run into two kinds of problems:

\begin{itemize}
\item \textit{Interleaving annotations} occur when an error doesn't end before
another one begins. Given the tokens $\tau_1 \tau_2 \tau_3$, and two errors
$\eta_1$ and
$\eta_2$ ranging over tokens $\tau_1$, $\tau_2$ and $\tau_2$ $\tau_3$ respectively, the
resulting markup will be confusing or outright impossible to read: $(\eta_1)\tau_1
(\eta_2)\tau_2(/\eta_1)\tau_3(/\eta_2)$. This is particularly a problem with XML,
since the specification\footnotemark explicitly disallows interleaving
markup.
\item \textit{Greedy annotation} covers tokens entirely uninvolved in the
\textit{``production''} of an error. If of the token sequence $\tau_1 \tau_2
\tau_3$ only $\tau_1$
and $\tau_3$ are erroneous,
marking the entire token sequence with an error annotation would falsely accuse
the
otherwise completely innocent token $\tau_2$.
\end{itemize}
\footnotetext{Located at \texttt{http://www.w3.org/TR/REC-xml/}}

Based on these assumptions, we decided to decouple the \textit{error markup}
from the \textit{corpus data}. Specifically, to our annotation method, the
errors and the text are two entirely different data structures. Every
particular error can reference tokens within the corpus using a \textbf{key},
similar to the way modern Relational Data Bases reference their data. This makes
the index more accessible and easier to maintain and eliminates both problems
above, because every single annotation can be completely independent from all
other annotations.

\subsection{Error Context}

Annotated learner corpora serve one primary purpose: categorizing and
cataloguing different kinds of errors that may occur in learner language. In
order for such data to be maximally useful, the error annotations have to be as
general as possible. We strive to improve the quality of the data by only
annotating erroneous tokens in a given error tag, thus not erroneously catching
'good' tokens in our error annotation. This, however, comes at a cost: if only
the ungrammatical token is identified as an error, its classification is no
longer justified from with in the error annotation itself. Consider the
following sentence:

\ex. Before she came, *She had going to the super market.

Here, the only erroneous token is \textit{going}, yet it is not by itself a
wrong word. This is where error context comes into play: since the error
annotation could not possibly tag "going" as a "badly formed verb tense", because, by
itself there is nothing wrong with it, we add "Before she came" to its
\textbf{error context} and apply the aforementioned type to the whole unit
consisting of erroneous token and error context. This error dependency mechanism
allows error annotations to be confined to a minimal space and still be
interpretable by themselves.

Another common pitfall for error taxonomies are ambiguous cases where the
annotator has to decide between several possible annotations. This may be
avoided by advising annotation of every conceived alternative in such cases,
creating overlapping markup. Typically, morphological errors, spelling errors,
word order errors, and other kinds of errors will form distinct levels that can
stack to a cumulative layer of errors on a single token. It is also interesting
to note that context/error pairs can form locking formations, where one error's
context can be another error, whose context will point to the original error in
turn.

\ex. *A mobile phones can be very useful.

Here, the determiner and the subject do not agree in number, but it is not clear
which one is wrong. There are two possible corrections, which would result
in the subject ending up as plural or singular, respectively. Therefore, we tag
two errors, a number agreement error on \textit{a} with context \textit{phones}, and
one on \textit{phones}, with context \textit{a}. Note that annotating only "a
phones" as an agreement error would not account for the two possible target
hypotheses. Such a scheme would have to invent a mechanism for defining multiple
target hypotheses\footnote{Although it is not clear whether our approach would
rid us of the necessity of such mechanisms. Word order mistakes pose a
significant problem to a one-target-hypothesis-per-error approach.}.

%In fact, in this way, the process of annotation could be partly automated, using an HPSG style feature compatibility check, adjusted for the purpose -  e.g. removing head dominance assumtpions (because in learner language, the head itself may be wrong), and instead taking a linear processing approach, to determine where grammatical errors are. 

\subsection{Error Taxonomy}
We adopted an error taxonomy to enrich our error identification annotations by
additional error classification. Error type is not dependent on the trigger
policy, any grammatical error type annotation scheme may be used here. To adjust
an annotation scheme to the error context paradigm, one must simply divide the
types among our three basic error type categories: Spelling, Grammar and
Context, as already described in \ref{sec:threeholies}.

For our test taxonomy, we used the taxonomy described in \cite{negrilloph} as a
base, and enhanced it by several concepts from the CLC annotation scheme, as
specified in \cite{CLC}. While we retained the general ideas and structure of
the aforementioned schemata, we aimed to improve their generality, and
systematization.

\begin{figure}
\centering
\includegraphics[scale=.2]{taxonomy}\label{fig:tax}
\caption{Error classification taxonomy.}
\end{figure}

We chose to follow a hierarchical setup, with some emphasis on typical decision
paths an annotator will have to make during the annotation process. The
hierarchical nature of our taxonomy made it a good fit for XML schemas, which
define relationships in similar ways.

In this taxonomy, "Error" is the root of all types of errors, with more and more
specific error classifications percolating down the tree. An annotator may
choose a less specific category in case they are unsure about a certain item.
Some errors can indeed not be classified at all, and will have to remain
annotated as the general category "Error." Since annotation of learner language
can be tricky at times, we chose to allow these kinds of underspecification of
error class in order to give an annotator the possibility to express an error
type more flexibly.

We also chose to alter a few concepts of the error taxonomies we based our own
on, specifically, we chose to eliminate certain kinds of error types to make the
annotation task less involved and more accurate.

\begin{itemize}
  \item \textit{Normative Errors.} Some normative concepts are present in almost
  all European languages, such as capitalization at the beginning of a sentence.
  We chose to ignore such errors, since they are not a sign of weakness in the
  target language but likely pure by chance mistakes.
  \item \textit{Style Errors.} Style is too soft a concept. Errors in style are
  notoriously hard to peg and reliably quantify. Moreover, our corpus
  contained data mostly from beginning English learners, where style errors are
  not as important, evident or relevant.
  \item \textit{Punctuation.} Punctuation mistakes are a very delicate concern:
  they can have tremendous effects on any given token string on both a syntactic
  and semantic level. They might change constituent boundaries, sentence
  boundaries, argument structures and many more things. Moreover, they tend to
  have cascading effects on the correctness of a given sentence. Again, because
  we were annotating a beginner's corpus, we chose to ignore these kinds of
  errors, since in beginner language punctuation often degrades to line noise
  very quickly.
\end{itemize}

We also included a "transfer" attribute to indicate that an error is an $L_1$
transfer error. This attribute was, however, not used in the annotation process,
since the annotators' knowledge of the Spanish language was insufficient to
make reliable judgements about it.

\subsection{Markup}

The source files for our corpus consisted of plain text files from the
NOCE-corpus (\cite{negrilloph})
of beginning to intermediate Spanish learners of English. Our corpus
format is defined in \ref{def:errorformat}

\begin{definition}\label{def:errorformat}

A corpus $\mathcal{C}=\pair{\mathcal{T},\mathcal{E}}$ is defined as a pair of
a set of tokens, $\mathcal{T}$ and a 
set of errors, $\mathcal{E}$. Every token $\tau_i$ is indexed with a unique
identifier subscript $i$\footnote{Note that the identifier may not consist entirely of
numbers, since it's XML type is \texttt{ID}, which demands identifiers to be
alphanumeric sequences. Furthermore, the identifiers need not adhere to any
particular order, as long as they are unique.}.
An error annotation is a tuple $\eta$ = $\pair{E,C,\theta,t,c}$, where
\begin{itemize}
  \item $E \subseteq \mathcal{T}$ is a nonempty set of indices of erroneous tokens
  \item $C \subseteq \mathcal{T}$ is a possibly empty set of context tokens
  \item $\theta$ is the type of the error,
  \item $s$ a string denoting an optional target hypothesis hint, and
  \item $c$ an optional comment.
\end{itemize}

The error type $\theta$ is defined as an ordered sequence of categories from the
taxonomy presented in \ref{sec:taxonomy}. Furthermore, $E \cap C = \emptyset$
for all $\eta\in\mathcal{E}$.

\end{definition}

This general data structure is translated to XML. A DTD ensures correct usage of
the markup. The authors also designed a graphical user interface based
annotation tool\footnotemark which hides the implementation details of the
corpus data from the user and enables a rapid annotation process. The user marks
tokens from the corpus as erroneous, assigns them a type, possibly an error
context and suggest a target hypothesis or records a comment entirely via the
interface.


\footnotetext{The tool is implemented in Haskell (\texttt{www.haskell.org}) on
top of the GTK  framework (\texttt{www.gtk.org}). It free and open source. The
sources to the tool and the XML markup format are given in appendix
\ref{appendix:github}.}

\subsection{Limitations}
Some elements, for example word order errors have enforced context, but it is hard to express via our schema, so we don't do it. 
No correct positions specified by word order errors - also hard to express. One could assign them a target position after an existing token, but then one must also keep track of other "reorderings" by multiple word order errors - I would like beer some to drink - here, both erroneous units - some and to drink should be placed between like and beer, but one must take care to order them correctly. Also, as our schema tags all possible errors, in the case of interchangeable errors (cf XXX), one cannot be certain if the error annotated is really the one, made by the learner, and not the other possible one. Most  importantly, there are sometimes more than one places where a word sequence can fit in.

Errors within a token cannot be analyzed – for example, agglutination, because a token is our smallest unit.

\section{Assessment}\label{sec:results}

Statistical inter-annotator agreement  measures are a common quality assessment
method used in corpus linguistics and related fields. Hereby, annotations made
on a particular data set by two or more annotators are compared using
quantitative methods.  \cite{artstein} give an overview of currently employed
methods.

While inter-annotator agreement measures have been applied successfully to
various corpus linguistic tasks, so far they have not found wide usage among
learner language annotation. We believe the current techniques  may not
applicable to this particular problem domain.

\subsection{Unitization and Multidimensional Markup}

Existing inter-annotator agreement measures all assume the presence of atomic
units in the corpus data, which are annotated by the annotators of a certain
data set.  The annotations \textit{over these units} are then used to calculate an
agreement coefficient. However, constituents of learner language, being fairly
diverse in nature, are not as easily contained in atomic units. Instead, a
common task in the annotation of learner language is \textit{delineating}
the extent of a certain mistake.

\cite{artstein} briefly discuss unitization and go on to note that it has thus far
not been exhaustively researched. Even more importantly, they explicitly comment
on the unknown status of the validity of the only inter-annotator agreement
measure in the corpus linguistic literature, $\alpha_U$, presented in
\cite{krip}.  Apart from being untested, $\alpha_U$ also seems to have
problems with overlapping markup both in the text, and between annotation sets
(both of which are frequent in learner language data.) The measure also assumes
annotation spans to be continuous, which is not the case in our data.

\subsection{Quantificational Analysis}

Facing these theoretical difficulties, we reached the conclusion that the only
viable way to define inter-annotator agreement over learner language data
we could improvise\footnotemark would still not yield interpretable results. In
order to quantify our analysis efforts, we analyzed the two annotation sets with
respect to their annotation's intersections.

\footnotetext{It would in theory be possible to regard tokens as units, and that
is indeed how our scheme is currently implemented. An agreement measure would
then interpret annotations locally on these units only. However, this method has
two major drawbacks: firstly, it does not account for more than one annotation
on a given unit, which happens frequently. Also, if one annotator marks a set of
tokens as an error consisting of more than one token, and the other annotator
marks these tokens with several errors of the same category (which can happen,
for example, with word order mistakes, as well as complex agreement mistakes),
this method would fail to account for the discrepancy.}

\begin{table}
  \centering
\begin{tabular}{r|c|c}\ref{tab:1}
  Total: & (905,933)& (98.69\%,101.74\%)\\\hline
  $io \wedge iy \wedge ir$:& 438 & 47.76 \%\\
  $so \wedge iy \wedge ir$:& 448 & 48.85 \%\\
  $io \wedge sy \wedge ir$:& 447 & 48.74 \%\\
  $io \wedge iy \wedge sr$:& 480 & 52.34 \%\\
  $so \wedge sy \wedge ir$:& 476 & 51.90 \%\\
  $io \wedge sy \wedge sr$:& 503 & 54.85 \%\\
  $so \wedge iy \wedge sr$:& 499 & 54.41 \%\\
  $so \wedge sy \wedge sr$:& 544 & 59.32 \%\\
  $     io \wedge iy$:& 486      & 52.99 \%\\
  $     so \wedge iy$:& 506      & 55.17 \%\\
  $     io \wedge sy$:& 509      & 55.50 \%\\
  $     so \wedge sy$:& 551      & 60.08 \%\\
  $     io \wedge ir$:& 525      & 57.25 \%\\
  $     so \wedge ir$:& 578      & 63.03 \%\\
  $     io \wedge sr$:& 623      & 67.93 \%\\
  $     so \wedge sr$:& 719      & 78.40 \%\\
  $          io$:& 629           & 68.59 \%\\
  $          so$:& 730           & 79.60 \%\\
\end{tabular}
\caption{Absolute amount of annotation overlap.}\label{tab:1}

\end{table}

Table \ref{tab:1} shows the total amount of corresponding data in the individual
markup. The table's labeling reads as follows: $s$ stands for a non-empty
intersection between the two data sets (or prefix relation for the error types)
and $i$ for equality.  $o,y,r$ denote the data type: $o$ stands for error
tOkens, $y$ for the error's tYpe, and $r$ for the errors context (or
\textit{tRigger}). Thus, $so\wedge sy\wedge ir$ is the number of all annotations
that have a partial overlap on the error tokens $E$, a prefix match on the error
type $\theta$, and identical error context.

\section{Conclusion}

\subsection{Possible Extensions}

After assessing the quality of our data, we reached
the conclusion that the error format described in \ref{sec:taxonomy}
might benefit from several refinements. Adding a field for part of speech tags
might contribute to the clarity of the data, as well as to its searchability.
The annotated corpus could be queried for erroneously placed verbs or
prepositions, for example.

Moreover, we came to the conclusion that defining the error context as a set of
tokens might be misleading or at least difficult to understand in case the error
context does not constitute one sequence, but several scattered sequences, such
as proper nouns or syntactic constituents. $C$ could therefore be turned into a
set of sequences of token indices.

Annotating a corpus with this schema is quite laborious, and borders on
impossibility without proper support from an annotation tool, such as the one we
had to devise. During the process, however, our tool was constantly improved
according to the annotators' ideas, and made the annotation process easier in
the process. Partial automation and other features might increase annotation
comfort even further.

The taxonomy presented in \label{taxonomy} proved to be a little unwieldy. In
particular, it did not clearly distinguish between annotating a target and
annotating an error type. The next subsection proposes a refinement that might
make the annotation process more precise.

A part of speech tagger could aid the annotators and enhance the corpus data
significantly. However, there are only a few reports on reliable part of speech
tagging for learner language. \cite{ludeling} presents one such approach.

\subsubsection{Towards an underspecification formalism for target hypotheses}

During the annotation process, we discovered that our taxonomy branches for
\textit{omission}, \textit{replacement}, and \textit{redundancy} could be turned
into a stub of a formalism for underspecification of target hypotheses. Instead
of giving a string for a target hypothesis, such a formalism would make it
possible to approximate the target and therefore allow for more flexibility in
the markup. Note, however, that these forms of omission, redundancy and
replacement differ from the ones included in the error taxonomy.

The taxonomy tries to account for \textit{what is wrong} with a given string of
text. A target hypothesis would try to make assumptions about \textit{how this
could be fixed}. Our categories in the error taxonomy suggesting manipulation of
the input text were explicitly designed to catch cases where a clear reason for
an error could not be found, and the syntactic environment of a given set of
tokens would require the text to be changed entirely. Such subcategorization
mistakes could be granted their own category and the target hypothesis could
account for the necessary steps in order to ensure grammaticality.

This would also allow for existing annotations to be combined with
\textit{generic instructions for correcting the input} and increase the
granularity of the data.

\appendix

\section{Annotation Manual}
  At any point, if you advance further down the taxonomy tree through a decision
  you make later on, do not mark the error as the more general category, but
  mark it as the more special category instead.

\subsection{Steps in the annotation process}
\begin{enumerate}
\item Look at each token in turn, assuming all the rest is grammatical.
\begin{itemize}
  \item If the token is not spelled correctly, record a spelling error.
  \item If an erroneous token $\tau_1$ is erroneous because it does not
  harmonize with a token $\tau_2$
  from its context,  mark $\tau_1$ as an error and $\tau_2$ as trigger.
  \item If the utterance is ungrammatical, i.e. it does not constitute a part of
  the language, mark it as a grammar mistake.
  \item If the sentence is utterable under a certain condition, but this
  condition is not met here, (i.e. forbidden by the general context of the
  text) mark it as a contextual error.
  \item Now, choose a subcategory from the branch you have chosen. Refer to the
  graph in \ref{fig:tax}.
  Try to be as specific as you can, otherwise, if not completely sure, choose
  the upper node. Try to see in what way a token can be changed, as to
  fit all the additional context. To find out which branch to choose, follow the
  instructions under \ref{branches}
  \item Repeat the process until all possible errors have been annotated, then
  move to the next token.
\end{itemize}

  \item If the context is too erroneous and this erroneous context part affects
  directly the immediate context of your token, try to find another solution,
  because otherwise you will be building error hypotheses on errors. If it doesn't
  (say, it is a spelling mistake), consider it as though it was correct while
  looking for a directly interacting token as trigger.

\item \textit{Local markup}: Keep it as local as possible, try to mark as
triggers tokens that are grammatically dependent on each other in normal
language production.

\item Go from tokens up to other units, such as phrases, and be minimalistic, unless this makes you lose important information.

\item Sometimes the error and/or the trigger will be a sequence. Take the
minimal unit you can that doesn't lead to loss of information. Always choose
the head of a constituent or compound word when it isn't the whole 
constituent that plays a role in error formation.

\ex.  The lovely, charming and amazing lady \textit{laugh} a lot 

Here, only take ``lady'' as a trigger. But in cases like \ref{granad}, your
trigger should be ``to study'', and not only ``to'', because otherwise you'd lose
information.  Maybe the learner doesn't have a problem with prepositions, but with infinitives.

\ex. I go for to study in Granada.\label{granad}

\item Avoid cascading and annotating errors based on corrected errors. Only look at the unaltered text.

\item Annotate every possibility. Since we do not apply cascading rules, one can use weighting algorithms to prefer one error to another in the case of concurrent errors:

\ex. \a.\textit{A} mobile \textbf{phones} can be very useful. 
\b. \textbf{A} mobile \textit{phones} can be very useful. 


\item Only correct when necessary, do not be tempted to rephrase a construction,
just because it would sound better or is more common, if there is nothing
explicitly wrong with it.

\item No style errors.

\item No punctuation errors

\item No capitalisation, only if it is something specific to English
(Spelling$\to$Register), such as, say, capitalisation of proper nouns or months.

\item If you have to choose between extremely similar possibilities, choose the
one whose target hypothesis would have the least impact on the text in terms of
phrasal constituents.

\end{enumerate}

\section{Branches in the taxonomy}\label{branches}

\subsection{Collocation vs. Replace/Omission/Redundant}

Verb and preposition collocations (i.e. where a certain verb forces a certain
collocation) are to be marked as Collocations, not as an erroneous word. Words
that occur with plural (\textit{one of}, \textit{between}) are also collocation
bound, and an error in agreement is thus a collocation error.

\subsection{Gerunds}

Gerunds are nominalized noun forms.  Place mistakes related to the formation of
gerunds in Morphology $\to$ Inflection, and errors related to the tense in which a
gerund ought (not) to occur in Grammar $\to$ Verbtense.

See also: Derivational Mistakes

\subsection{Morphology}

Morphological negation and plural formation mistakes often occur in English
learner language. We tag them as Morphology and Morphology $\to$ Inflection instead
of choosing a more specific category, but might extend our scheme in the future.

\subsection{Derivational Mistakes}

If a word is wrongly derived from another, (nominalization, suffixation,
prefixation, etc\ldots) tag it as a morphological derivation mistake.

\subsection{Animate/Inanimate}

Is not covered by the scheme right now. Use Replace $\to$ Pronoun or similar. Might
be added in the future.

\subsection{Subject, Object, Predicate vs Pronoun, noun, verb.}

If a whole constituent is missing or wrong, use the constituent classes. If it's
one word out of the constituent, use the part of speech classes. Again: try to
be as general as possible.

\subsection{Number Agreement vs. Redundancy}

\ex. a dogs

\textit{a} is redundant, but because of number agreement. So you must annotate both,
creating a circular dependency, where an erroneous \textit{dogs} has as error
context \textit{a} and an erroneous \textit{a} has as context \textit{docs}

\subsection{Compound nouns}

If a compound noun of two or more words is erroneous, annotate only the head noun, carrying the mistake.

\subsection{Sequences}
For both triggers and erroneous tokens, it is best to use a single token. In
case of a phrase, only take the keyword and at most any relevant information
bearing token. In such a case, do not care about order.

Still, fixed expressions, such as  Collocations or Idiom can be counted as one
unit. In such a case, annotate the whole phrase, even if it contains erroneous
tokens. Only do not annotate the target if it is within the idiom.


\subsection{Word order}
\begin{itemize}
\item \textit{Internal word order.} Words should be shuffled within the marked sequence. 
\ex. Not only I must study $\to$ Not only must I study

\item \textit{External word order.} Specified word needs to be moved somewhere
else, its elements moving together as a unit.
\ex. I the river like $\to$ I like the river

\end{itemize}

\subsection{Verb Tense}

If the verb tense is wrongly formed, (\textit{he had having a nice time}), it's
a grammatical error. Also, if there is a typical trigger word for verb tenses
(such as \textit{since}, \textit{before}, \textit{if}) and the tense is wrong,
it's a grammatical error. If the tense doesn't fit because of the broader
context, it's a contextual mistake.

\section{Source Code and XML Data}\label{appendix:github}

The complete code, as well as an up-to-date version of this paper can be found
online at \texttt{www.github.com/adimit/ll-annotation}. The annotation schema's
DTD is \texttt{AnnotationScheme.dtd} in the root directory of the repository.

\bibliography{bib}

\end{document}
